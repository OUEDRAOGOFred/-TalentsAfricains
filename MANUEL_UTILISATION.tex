\documentclass[12pt,a4paper]{article}
\usepackage[utf8]{inputenc}
\usepackage[french]{babel}
\usepackage[T1]{fontenc}
\usepackage{geometry}
\usepackage{graphicx}
\usepackage{xcolor}
\usepackage{hyperref}
\usepackage{enumitem}
\usepackage{tcolorbox}
\usepackage{fancyhdr}
\usepackage{titlesec}
\usepackage{listings}
\usepackage{fontawesome5}

\geometry{margin=2.5cm}

% Couleurs personnalisées
\definecolor{primarycolor}{RGB}{45,134,89}
\definecolor{secondarycolor}{RGB}{212,166,97}
\definecolor{darkcolor}{RGB}{26,26,46}
\definecolor{lightgray}{RGB}{245,245,245}

% Configuration des liens
\hypersetup{
    colorlinks=true,
    linkcolor=primarycolor,
    urlcolor=primarycolor,
    citecolor=primarycolor
}

% En-tête et pied de page
\pagestyle{fancy}
\fancyhf{}
\fancyhead[L]{\textcolor{primarycolor}{\textbf{TalentsAfricains}}}
\fancyhead[R]{\textcolor{gray}{Manuel d'Utilisation}}
\fancyfoot[C]{\thepage}
\renewcommand{\headrulewidth}{0.5pt}
\renewcommand{\footrulewidth}{0.5pt}

% Style des titres
\titleformat{\section}
  {\normalfont\Large\bfseries\color{primarycolor}}
  {\thesection}{1em}{}
  
\titleformat{\subsection}
  {\normalfont\large\bfseries\color{secondarycolor}}
  {\thesubsection}{1em}{}

% Boîtes d'information
\newtcolorbox{infobox}[1]{
  colback=lightgray,
  colframe=primarycolor,
  fonttitle=\bfseries,
  title=#1,
  arc=3mm
}

\newtcolorbox{warningbox}[1]{
  colback=yellow!10,
  colframe=orange,
  fonttitle=\bfseries,
  title=#1,
  arc=3mm
}

\newtcolorbox{credentialbox}{
  colback=blue!5,
  colframe=blue!60,
  fonttitle=\bfseries,
  title=\faKey\ Identifiants Administrateur,
  arc=3mm,
  boxrule=1.5pt
}

\begin{document}

% Page de titre
\begin{titlepage}
    \centering
    \vspace*{2cm}
    
    {\Huge\bfseries\color{primarycolor} TalentsAfricains\par}
    \vspace{0.5cm}
    {\Large\color{secondarycolor} Plateforme de Mise en Valeur des Talents Africains\par}
    
    \vspace{2cm}
    
    {\LARGE\bfseries Manuel d'Utilisation\par}
    \vspace{0.5cm}
    {\large Version 1.0\par}
    
    \vspace{3cm}
    
    \begin{tcolorbox}[colback=lightgray, colframe=primarycolor, width=0.8\textwidth, arc=5mm]
        \centering
        \Large\textbf{Guide Complet}\\[0.3cm]
        \normalsize
        Installation • Configuration • Utilisation • Administration
    \end{tcolorbox}
    
    \vfill
    
    {\large Développé par\par}
    {\Large\bfseries\color{primarycolor} Freddy OUEDRAOGO\par}
    \vspace{0.3cm}
    {\normalsize \today\par}
\end{titlepage}

\newpage
\tableofcontents
\newpage

% ===================================
% SECTION 1: PRÉSENTATION
% ===================================
\section{Présentation de la Plateforme}

\subsection{Vue d'ensemble}

\textbf{TalentsAfricains} est une plateforme web moderne et innovante conçue pour mettre en lumière les talents africains à travers leurs projets créatifs, entrepreneuriaux et innovants. La plateforme offre un espace où les créateurs peuvent partager leurs réalisations et se connecter avec une communauté engagée.

\subsection{Fonctionnalités Principales}

\begin{itemize}[leftmargin=*]
    \item \textbf{\faGlobe\ Découverte de projets} : Parcourir et filtrer les projets par catégorie et localisation
    \item \textbf{\faUpload\ Publication de projets} : Partager vos créations avec galerie d'images
    \item \textbf{\faHeart\ Interactions sociales} : Liker et commenter les projets
    \item \textbf{\faUser\ Profils utilisateurs} : Gérer votre compte et vos projets
    \item \textbf{\faUserShield\ Panel administrateur} : Gestion complète de la plateforme
    \item \textbf{\faSearch\ Recherche avancée} : Trouver des projets par mots-clés
    \item \textbf{\faFilter\ Filtres intelligents} : Tri par popularité, date, catégorie
\end{itemize}

\subsection{Technologies Utilisées}

\begin{infobox}{Stack Technique}
\begin{itemize}[leftmargin=*]
    \item \textbf{Frontend} : React 18.2, Vite 5.4, CSS3 moderne
    \item \textbf{Backend} : Node.js 22.15, Express.js 4.18
    \item \textbf{Base de données} : MySQL 8.0
    \item \textbf{Authentification} : JWT (JSON Web Tokens)
    \item \textbf{Sécurité} : bcrypt pour le hachage des mots de passe
    \item \textbf{Upload} : Multer pour la gestion des fichiers
\end{itemize}
\end{infobox}

% ===================================
% SECTION 2: INSTALLATION
% ===================================
\newpage
\section{Installation et Configuration}

\subsection{Prérequis Système}

Avant d'installer la plateforme, assurez-vous que votre système dispose de :

\begin{itemize}[leftmargin=*]
    \item \textbf{Node.js} : Version 18.x ou supérieure
    \item \textbf{MySQL} : Version 8.0 ou supérieure
    \item \textbf{Git} : Pour cloner le dépôt
    \item \textbf{npm} : Gestionnaire de paquets Node.js
\end{itemize}

\subsection{Installation pas à pas}

\subsubsection{1. Cloner le dépôt}

\begin{tcolorbox}[colback=darkcolor!5, colframe=darkcolor, fonttitle=\ttfamily]
git clone https://github.com/OUEDRAOGOFred/-TalentsAfricains.git\\
cd TalentsAfricains
\end{tcolorbox}

\subsubsection{2. Configuration de la base de données}

\begin{enumerate}[leftmargin=*]
    \item Créer la base de données MySQL :
    \begin{tcolorbox}[colback=darkcolor!5, colframe=darkcolor, fonttitle=\ttfamily]
    mysql -u root -p < database/talentsafricains.sql
    \end{tcolorbox}
    
    \item Créer le compte administrateur :
    \begin{tcolorbox}[colback=darkcolor!5, colframe=darkcolor, fonttitle=\ttfamily]
    mysql -u root -p talentsafricains < database/add\_admin.sql
    \end{tcolorbox}
\end{enumerate}

\subsubsection{3. Configuration du Backend}

\begin{enumerate}[leftmargin=*]
    \item Naviguer vers le dossier backend :
    \begin{tcolorbox}[colback=darkcolor!5, colframe=darkcolor, fonttitle=\ttfamily]
    cd backend
    \end{tcolorbox}
    
    \item Installer les dépendances :
    \begin{tcolorbox}[colback=darkcolor!5, colframe=darkcolor, fonttitle=\ttfamily]
    npm install
    \end{tcolorbox}
    
    \item Créer le fichier \texttt{.env} :
    \begin{tcolorbox}[colback=darkcolor!5, colframe=darkcolor, fonttitle=\ttfamily, fontsize=\small]
    DB\_HOST=localhost\\
    DB\_USER=root\\
    DB\_PASSWORD=votre\_mot\_de\_passe\\
    DB\_NAME=talentsafricains\\
    JWT\_SECRET=votre\_secret\_jwt\_tres\_securise\\
    PORT=5000
    \end{tcolorbox}
    
    \item Démarrer le serveur backend :
    \begin{tcolorbox}[colback=darkcolor!5, colframe=darkcolor, fonttitle=\ttfamily]
    npm run dev
    \end{tcolorbox}
\end{enumerate}

\subsubsection{4. Configuration du Frontend}

\begin{enumerate}[leftmargin=*]
    \item Naviguer vers le dossier frontend :
    \begin{tcolorbox}[colback=darkcolor!5, colframe=darkcolor, fonttitle=\ttfamily]
    cd ../frontend
    \end{tcolorbox}
    
    \item Installer les dépendances :
    \begin{tcolorbox}[colback=darkcolor!5, colframe=darkcolor, fonttitle=\ttfamily]
    npm install
    \end{tcolorbox}
    
    \item Créer le fichier \texttt{.env} :
    \begin{tcolorbox}[colback=darkcolor!5, colframe=darkcolor, fonttitle=\ttfamily]
    VITE\_API\_URL=http://localhost:5000/api
    \end{tcolorbox}
    
    \item Démarrer le serveur frontend :
    \begin{tcolorbox}[colback=darkcolor!5, colframe=darkcolor, fonttitle=\ttfamily]
    npm run dev
    \end{tcolorbox}
\end{enumerate}

\subsection{Accès à la Plateforme}

Une fois l'installation terminée, accédez à :
\begin{itemize}[leftmargin=*]
    \item \textbf{Frontend} : \url{http://localhost:3000}
    \item \textbf{Backend API} : \url{http://localhost:5000}
\end{itemize}

% ===================================
% SECTION 3: IDENTIFIANTS
% ===================================
\newpage
\section{Identifiants de Connexion}

\begin{credentialbox}
\Large
\textbf{Email :} \texttt{admin@talentsafricains.com}\\[0.3cm]
\textbf{Mot de passe :} \texttt{Password123!}\\[0.5cm]

\normalsize
\textcolor{red}{\faExclamationTriangle\ \textbf{IMPORTANT :} Changez ce mot de passe après la première connexion pour des raisons de sécurité !}
\end{credentialbox}

\subsection{Comptes de Test}

Pour tester les fonctionnalités utilisateur standard :

\begin{tcolorbox}[colback=lightgray, colframe=secondarycolor]
\textbf{Porteur de projet 1}\\
Email : \texttt{amina.diallo@example.com}\\
Mot de passe : \texttt{Password123!}\\[0.3cm]

\textbf{Porteur de projet 2}\\
Email : \texttt{kwame.mensah@example.com}\\
Mot de passe : \texttt{Password123!}\\[0.3cm]

\textbf{Visiteur}\\
Email : \texttt{fatou.kone@example.com}\\
Mot de passe : \texttt{Password123!}
\end{tcolorbox}

% ===================================
% SECTION 4: GUIDE UTILISATEUR
% ===================================
\newpage
\section{Guide Utilisateur}

\subsection{Page d'Accueil}

La page d'accueil présente :
\begin{itemize}[leftmargin=*]
    \item \textbf{Statistiques en temps réel} : Nombre de projets, porteurs, visiteurs et commentaires
    \item \textbf{Projets populaires} : Les projets les plus likés
    \item \textbf{Projets récents} : Les dernières publications
    \item \textbf{Catégories} : Navigation rapide par domaine
\end{itemize}

\subsection{Inscription et Connexion}

\subsubsection{S'inscrire}
\begin{enumerate}[leftmargin=*]
    \item Cliquer sur \textbf{« Connexion »} dans le menu
    \item Sélectionner l'onglet \textbf{« S'inscrire »}
    \item Remplir le formulaire :
    \begin{itemize}
        \item Prénom et Nom
        \item Email (unique)
        \item Mot de passe (minimum 6 caractères)
        \item Téléphone et Localisation
        \item Biographie
        \item Choisir le rôle : Porteur de projet ou Visiteur
    \end{itemize}
    \item Cliquer sur \textbf{« S'inscrire »}
\end{enumerate}

\subsubsection{Se connecter}
\begin{enumerate}[leftmargin=*]
    \item Cliquer sur \textbf{« Connexion »}
    \item Entrer votre email et mot de passe
    \item Cliquer sur \textbf{« Se connecter »}
\end{enumerate}

\subsection{Découvrir les Projets}

\subsubsection{Navigation}
\begin{itemize}[leftmargin=*]
    \item Accéder à la page \textbf{« Découvrir »} depuis le menu
    \item Utiliser la barre de recherche pour trouver des projets
    \item Filtrer par :
    \begin{itemize}
        \item Catégorie (Technologie, Art, Entrepreneuriat, etc.)
        \item Localisation
        \item Tri (Récent, Populaire, Ancien)
    \end{itemize}
\end{itemize}

\subsubsection{Consulter un Projet}
\begin{itemize}[leftmargin=*]
    \item Cliquer sur une carte projet
    \item Consulter les détails : description, images, auteur
    \item Voir les statistiques : likes, commentaires, vues
    \item Liker le projet (\faHeart)
    \item Laisser un commentaire
\end{itemize}

\subsection{Publier un Projet}

\begin{warningbox}{\faInfoCircle\ Prérequis}
Vous devez être connecté avec un compte \textbf{Porteur de projet} pour publier.
\end{warningbox}

\subsubsection{Étapes de publication}
\begin{enumerate}[leftmargin=*]
    \item Cliquer sur \textbf{« Ajouter un Projet »} dans le menu
    \item Remplir le formulaire :
    \begin{itemize}
        \item \textbf{Titre} : Nom du projet (obligatoire)
        \item \textbf{Description} : Présentation détaillée
        \item \textbf{Catégorie} : Choisir parmi 8 catégories
        \item \textbf{Localisation} : Pays/Ville
        \item \textbf{Lien externe} : URL du projet (optionnel)
        \item \textbf{Image principale} : Photo de couverture (JPG, PNG)
        \item \textbf{Galerie} : Jusqu'à 5 images supplémentaires
    \end{itemize}
    \item Cliquer sur \textbf{« Publier le Projet »}
\end{enumerate}

\subsection{Gérer son Profil}

\subsubsection{Accéder au profil}
\begin{itemize}[leftmargin=*]
    \item Cliquer sur \textbf{« Mon profil »} dans le menu
    \item Consulter vos informations
    \item Voir la liste de vos projets publiés
\end{itemize}

\subsubsection{Modifier le profil}
\begin{itemize}[leftmargin=*]
    \item Cliquer sur \textbf{« Modifier le Profil »}
    \item Mettre à jour :
    \begin{itemize}
        \item Photo de profil
        \item Prénom, Nom
        \item Téléphone, Localisation
        \item Biographie
        \item Mot de passe (optionnel)
    \end{itemize}
    \item Cliquer sur \textbf{« Enregistrer les Modifications »}
\end{itemize}

\subsubsection{Gérer ses projets}
\begin{itemize}[leftmargin=*]
    \item Voir tous vos projets dans l'onglet \textbf{« Mes Projets »}
    \item Modifier un projet (titre, description, images)
    \item Supprimer un projet (action irréversible)
\end{itemize}

% ===================================
% SECTION 5: PANEL ADMINISTRATEUR
% ===================================
\newpage
\section{Panel Administrateur}

\subsection{Accès au Dashboard}

\begin{enumerate}[leftmargin=*]
    \item Se connecter avec le compte administrateur
    \item Cliquer sur \textbf{« \faUserShield\ Dashboard Admin »} dans le menu
\end{enumerate}

\subsection{Vue d'ensemble}

Le tableau de bord affiche :

\begin{tcolorbox}[colback=primarycolor!10, colframe=primarycolor, title=\faChartLine\ Statistiques Globales]
\begin{itemize}[leftmargin=*]
    \item Nombre total de projets publiés
    \item Nombre d'utilisateurs inscrits
    \item Nombre de commentaires
    \item Nombre de likes
\end{itemize}
\end{tcolorbox}

\subsection{Gestion des Projets}

\subsubsection{Tableau des projets}
\begin{itemize}[leftmargin=*]
    \item Liste complète de tous les projets
    \item Informations affichées :
    \begin{itemize}
        \item Titre et auteur
        \item Catégorie et localisation
        \item Date de publication
        \item Statistiques (likes, vues, commentaires)
        \item Statut (Publié, En attente, Rejeté)
    \end{itemize}
\end{itemize}

\subsubsection{Actions disponibles}
\begin{itemize}[leftmargin=*]
    \item \textbf{\faEye\ Voir} : Consulter le projet en détail
    \item \textbf{\faEdit\ Modifier} : Éditer le contenu
    \item \textbf{\faTrash\ Supprimer} : Retirer définitivement
    \item \textbf{\faCheck\ Approuver} : Valider la publication
    \item \textbf{\faTimes\ Rejeter} : Refuser le projet
\end{itemize}

\subsection{Gestion des Utilisateurs}

\subsubsection{Tableau des utilisateurs}
\begin{itemize}[leftmargin=*]
    \item Liste de tous les comptes
    \item Informations affichées :
    \begin{itemize}
        \item Nom complet
        \item Email et téléphone
        \item Rôle (Admin, Porteur, Visiteur)
        \item Localisation
        \item Date d'inscription
        \item Nombre de projets publiés
    \end{itemize}
\end{itemize}

\subsubsection{Actions disponibles}
\begin{itemize}[leftmargin=*]
    \item \textbf{\faEye\ Voir profil} : Consulter le profil complet
    \item \textbf{\faUserShield\ Changer rôle} : Modifier les permissions
    \item \textbf{\faBan\ Suspendre} : Désactiver temporairement le compte
    \item \textbf{\faTrash\ Supprimer} : Supprimer définitivement l'utilisateur
\end{itemize}

\subsection{Modération des Commentaires}

\begin{itemize}[leftmargin=*]
    \item Consulter tous les commentaires publiés
    \item Supprimer les commentaires inappropriés
    \item Bloquer les utilisateurs abusifs
    \item Voir l'historique des interactions
\end{itemize}

\subsection{Rapports et Statistiques}

\subsubsection{Rapports disponibles}
\begin{itemize}[leftmargin=*]
    \item \textbf{Évolution des inscriptions} : Graphique temporel
    \item \textbf{Projets par catégorie} : Répartition
    \item \textbf{Projets les plus populaires} : Top 10
    \item \textbf{Utilisateurs les plus actifs} : Classement
    \item \textbf{Activité quotidienne} : Likes et commentaires
\end{itemize}

\subsubsection{Export de données}
\begin{itemize}[leftmargin=*]
    \item Exporter la liste des utilisateurs (CSV)
    \item Exporter la liste des projets (CSV)
    \item Générer des rapports PDF
\end{itemize}

% ===================================
% SECTION 6: FONCTIONNALITÉS AVANCÉES
% ===================================
\newpage
\section{Fonctionnalités Avancées}

\subsection{Système de Likes}

\begin{itemize}[leftmargin=*]
    \item Un utilisateur peut liker un projet une seule fois
    \item Le like peut être retiré en cliquant à nouveau
    \item Le compteur de likes est mis à jour en temps réel
    \item Les projets sont classés par popularité basée sur les likes
\end{itemize}

\subsection{Système de Commentaires}

\subsubsection{Publier un commentaire}
\begin{enumerate}[leftmargin=*]
    \item Ouvrir un projet
    \item Faire défiler jusqu'à la section commentaires
    \item Écrire votre commentaire dans la zone de texte
    \item Cliquer sur \textbf{« Publier »}
\end{enumerate}

\subsubsection{Modération}
\begin{itemize}[leftmargin=*]
    \item Les utilisateurs peuvent supprimer leurs propres commentaires
    \item Les administrateurs peuvent supprimer tous les commentaires
    \item Les commentaires sont horodatés
\end{itemize}

\subsection{Upload de Fichiers}

\begin{infobox}{Formats Acceptés}
\begin{itemize}[leftmargin=*]
    \item \textbf{Images} : JPG, JPEG, PNG, GIF
    \item \textbf{Taille maximale} : 5 MB par image
    \item \textbf{Galerie} : Jusqu'à 5 images par projet
\end{itemize}
\end{infobox}

\subsection{Recherche et Filtres}

\subsubsection{Recherche par mots-clés}
\begin{itemize}[leftmargin=*]
    \item Recherche dans les titres
    \item Recherche dans les descriptions
    \item Recherche insensible à la casse
    \item Résultats mis en surbrillance
\end{itemize}

\subsubsection{Filtres disponibles}
\begin{itemize}[leftmargin=*]
    \item \textbf{Catégorie} : 8 catégories disponibles
    \item \textbf{Localisation} : Par pays ou ville
    \item \textbf{Tri} :
    \begin{itemize}
        \item Plus récents
        \item Plus populaires (par likes)
        \item Plus anciens
    \end{itemize}
\end{itemize}

\subsection{Notifications}

\begin{itemize}[leftmargin=*]
    \item Notification lors d'un nouveau like sur votre projet
    \item Notification lors d'un nouveau commentaire
    \item Notification d'approbation/rejet par l'admin
    \item Notification de modification de compte
\end{itemize}

% ===================================
% SECTION 7: SÉCURITÉ
% ===================================
\newpage
\section{Sécurité et Confidentialité}

\subsection{Authentification}

\begin{tcolorbox}[colback=blue!5, colframe=blue!60, title=\faLock\ Mesures de Sécurité]
\begin{itemize}[leftmargin=*]
    \item \textbf{Hachage des mots de passe} : bcrypt avec salt de 10 rounds
    \item \textbf{Tokens JWT} : Expiration après 24 heures
    \item \textbf{Protection CSRF} : Tokens anti-falsification
    \item \textbf{Validation des entrées} : Sanitization côté serveur
    \item \textbf{HTTPS recommandé} : Pour le déploiement en production
\end{itemize}
\end{tcolorbox}

\subsection{Permissions et Rôles}

\begin{table}[h]
\centering
\begin{tabular}{|l|c|c|c|}
\hline
\textbf{Action} & \textbf{Visiteur} & \textbf{Porteur} & \textbf{Admin} \\ \hline
Voir les projets & \checkmark & \checkmark & \checkmark \\ \hline
Liker/Commenter & \checkmark & \checkmark & \checkmark \\ \hline
Publier un projet & \texttimes & \checkmark & \checkmark \\ \hline
Modifier ses projets & \texttimes & \checkmark & \checkmark \\ \hline
Gérer les utilisateurs & \texttimes & \texttimes & \checkmark \\ \hline
Modérer le contenu & \texttimes & \texttimes & \checkmark \\ \hline
Accès au dashboard & \texttimes & \texttimes & \checkmark \\ \hline
\end{tabular}
\caption{Matrice des permissions par rôle}
\end{table}

\subsection{Protection des Données}

\begin{itemize}[leftmargin=*]
    \item Les mots de passe ne sont jamais stockés en clair
    \item Les emails sont validés et uniques
    \item Les uploads sont vérifiés (type MIME, taille)
    \item Les injections SQL sont prévenues (prepared statements)
    \item Les XSS sont neutralisées (échappement HTML)
\end{itemize}

\subsection{Bonnes Pratiques}

\begin{warningbox}{\faExclamationTriangle\ Recommandations}
\begin{itemize}[leftmargin=*]
    \item Changer le mot de passe administrateur par défaut
    \item Utiliser des mots de passe complexes (min. 12 caractères)
    \item Activer HTTPS en production
    \item Faire des sauvegardes régulières de la base de données
    \item Mettre à jour régulièrement les dépendances npm
    \item Surveiller les logs pour détecter les activités suspectes
\end{itemize}
\end{warningbox}

% ===================================
% SECTION 8: MAINTENANCE
% ===================================
\newpage
\section{Maintenance et Support}

\subsection{Sauvegardes}

\subsubsection{Sauvegarde de la base de données}
\begin{tcolorbox}[colback=darkcolor!5, colframe=darkcolor, fonttitle=\ttfamily]
mysqldump -u root -p talentsafricains > backup\_\$(date +\%Y\%m\%d).sql
\end{tcolorbox}

\subsubsection{Restauration}
\begin{tcolorbox}[colback=darkcolor!5, colframe=darkcolor, fonttitle=\ttfamily]
mysql -u root -p talentsafricains < backup\_20250131.sql
\end{tcolorbox}

\subsection{Mises à Jour}

\subsubsection{Backend}
\begin{tcolorbox}[colback=darkcolor!5, colframe=darkcolor, fonttitle=\ttfamily]
cd backend\\
npm update\\
npm audit fix
\end{tcolorbox}

\subsubsection{Frontend}
\begin{tcolorbox}[colback=darkcolor!5, colframe=darkcolor, fonttitle=\ttfamily]
cd frontend\\
npm update\\
npm audit fix
\end{tcolorbox}

\subsection{Logs et Débogage}

\begin{itemize}[leftmargin=*]
    \item \textbf{Logs Backend} : Consultez la console du terminal backend
    \item \textbf{Logs Frontend} : Utilisez la console du navigateur (F12)
    \item \textbf{Logs MySQL} : Consultez les logs d'erreur MySQL
    \item \textbf{Mode développement} : Activez le mode verbose pour plus de détails
\end{itemize}

\subsection{Problèmes Courants}

\subsubsection{Le serveur ne démarre pas}
\begin{itemize}[leftmargin=*]
    \item Vérifiez que MySQL est démarré
    \item Vérifiez les identifiants dans le fichier \texttt{.env}
    \item Assurez-vous que le port 5000 n'est pas déjà utilisé
    \item Vérifiez les logs pour identifier l'erreur
\end{itemize}

\subsubsection{Les images ne s'affichent pas}
\begin{itemize}[leftmargin=*]
    \item Vérifiez que le dossier \texttt{uploads/} existe
    \item Vérifiez les permissions du dossier (lecture/écriture)
    \item Vérifiez l'URL de l'API dans le fichier \texttt{.env}
\end{itemize}

\subsubsection{Impossible de se connecter}
\begin{itemize}[leftmargin=*]
    \item Vérifiez que l'email et le mot de passe sont corrects
    \item Vérifiez que le compte n'est pas suspendu
    \item Effacez le cache du navigateur
    \item Vérifiez que le backend est accessible
\end{itemize}

% ===================================
% SECTION 9: DÉPLOIEMENT
% ===================================
\newpage
\section{Déploiement en Production}

\subsection{Configuration Serveur}

\subsubsection{Serveur recommandé}
\begin{itemize}[leftmargin=*]
    \item \textbf{OS} : Ubuntu 20.04 LTS ou supérieur
    \item \textbf{RAM} : Minimum 2 GB (4 GB recommandé)
    \item \textbf{CPU} : 2 vCPU minimum
    \item \textbf{Disque} : 20 GB SSD minimum
    \item \textbf{Node.js} : Version 18.x ou 20.x
    \item \textbf{MySQL} : Version 8.0
\end{itemize}

\subsection{Configuration Nginx}

Exemple de configuration pour Nginx :

\begin{tcolorbox}[colback=darkcolor!5, colframe=darkcolor, fonttitle=\ttfamily, fontsize=\small]
server \{\\
\quad listen 80;\\
\quad server\_name talentsafricains.com;\\
\\
\quad location / \{\\
\quad\quad proxy\_pass http://localhost:3000;\\
\quad\quad proxy\_http\_version 1.1;\\
\quad\quad proxy\_set\_header Upgrade \$http\_upgrade;\\
\quad\quad proxy\_set\_header Connection 'upgrade';\\
\quad\quad proxy\_set\_header Host \$host;\\
\quad\quad proxy\_cache\_bypass \$http\_upgrade;\\
\quad \}\\
\\
\quad location /api \{\\
\quad\quad proxy\_pass http://localhost:5000;\\
\quad \}\\
\}
\end{tcolorbox}

\subsection{Process Manager (PM2)}

Pour maintenir le backend en ligne :

\begin{tcolorbox}[colback=darkcolor!5, colframe=darkcolor, fonttitle=\ttfamily]
npm install -g pm2\\
pm2 start src/server.js --name talentsafricains-api\\
pm2 save\\
pm2 startup
\end{tcolorbox}

\subsection{Variables d'Environnement Production}

\begin{tcolorbox}[colback=yellow!10, colframe=orange, fonttitle=\ttfamily, fontsize=\small]
NODE\_ENV=production\\
DB\_HOST=localhost\\
DB\_USER=talentsafricains\\
DB\_PASSWORD=mot\_de\_passe\_securise\\
DB\_NAME=talentsafricains\\
JWT\_SECRET=secret\_tres\_securise\_aleatoire\\
PORT=5000\\
FRONTEND\_URL=https://talentsafricains.com
\end{tcolorbox}

% ===================================
% SECTION 10: SUPPORT
% ===================================
\newpage
\section{Support et Contact}

\subsection{Documentation}

\begin{itemize}[leftmargin=*]
    \item \textbf{README.md} : Instructions rapides d'installation
    \item \textbf{QUICKSTART.md} : Guide de démarrage rapide
    \item \textbf{ADMIN\_CREDENTIALS.md} : Identifiants administrateur
    \item \textbf{Ce manuel} : Documentation complète
\end{itemize}

\subsection{Dépôt GitHub}

\begin{tcolorbox}[colback=primarycolor!10, colframe=primarycolor]
\centering
\faGithub\ \url{https://github.com/OUEDRAOGOFred/-TalentsAfricains}
\end{tcolorbox}

\subsection{Contact Développeur}

Pour toute question technique ou demande de support :

\begin{tcolorbox}[colback=secondarycolor!10, colframe=secondarycolor, title=\faUser\ Développeur]
\textbf{Nom :} Freddy OUEDRAOGO\\
\textbf{Email :} contact@freddyouedraogo.dev\\
\textbf{GitHub :} @OUEDRAOGOFred
\end{tcolorbox}

\subsection{Licence et Utilisation}

Cette plateforme a été développée dans un cadre professionnel. Tous droits réservés.

\begin{itemize}[leftmargin=*]
    \item Utilisation autorisée pour le client
    \item Modification et personnalisation permises
    \item Redistribution non autorisée sans accord préalable
\end{itemize}

% ===================================
% ANNEXES
% ===================================
\newpage
\section*{Annexes}
\addcontentsline{toc}{section}{Annexes}

\subsection*{Annexe A : Catégories de Projets}

\begin{table}[h]
\centering
\begin{tabular}{|l|l|}
\hline
\textbf{Catégorie} & \textbf{Description} \\ \hline
Technologie & Applications, logiciels, IA, blockchain \\ \hline
Art & Peinture, sculpture, photographie, design \\ \hline
Entrepreneuriat & Startups, business plans, innovations \\ \hline
Innovation & Inventions, solutions nouvelles \\ \hline
Éducation & Plateformes d'apprentissage, formations \\ \hline
Santé & Applications médicales, bien-être \\ \hline
Agriculture & AgriTech, solutions agricoles \\ \hline
Autre & Projets divers non catégorisés \\ \hline
\end{tabular}
\caption{Liste des catégories disponibles}
\end{table}

\subsection*{Annexe B : Structure de la Base de Données}

\textbf{Tables principales :}

\begin{itemize}[leftmargin=*]
    \item \texttt{users} : Comptes utilisateurs
    \item \texttt{projects} : Projets publiés
    \item \texttt{comments} : Commentaires sur les projets
    \item \texttt{likes} : Likes des utilisateurs
\end{itemize}

\subsection*{Annexe C : API Endpoints}

\textbf{Authentification :}
\begin{itemize}[leftmargin=*]
    \item \texttt{POST /api/auth/register} : Inscription
    \item \texttt{POST /api/auth/login} : Connexion
    \item \texttt{GET /api/auth/me} : Profil actuel
\end{itemize}

\textbf{Projets :}
\begin{itemize}[leftmargin=*]
    \item \texttt{GET /api/projects} : Liste des projets
    \item \texttt{GET /api/projects/:id} : Détails d'un projet
    \item \texttt{POST /api/projects} : Créer un projet
    \item \texttt{PUT /api/projects/:id} : Modifier un projet
    \item \texttt{DELETE /api/projects/:id} : Supprimer un projet
\end{itemize}

\textbf{Interactions :}
\begin{itemize}[leftmargin=*]
    \item \texttt{POST /api/likes/:projectId} : Liker/Unliker
    \item \texttt{POST /api/comments/:projectId} : Commenter
    \item \texttt{DELETE /api/comments/:id} : Supprimer un commentaire
\end{itemize}

\textbf{Administration :}
\begin{itemize}[leftmargin=*]
    \item \texttt{GET /api/admin/statistics} : Statistiques globales
    \item \texttt{GET /api/admin/users} : Liste des utilisateurs
    \item \texttt{GET /api/admin/projects} : Tous les projets
    \item \texttt{PUT /api/admin/users/:id/role} : Changer le rôle
\end{itemize}

% ===================================
% PAGE FINALE
% ===================================
\newpage
\begin{center}
\vspace*{5cm}

{\Huge\bfseries\color{primarycolor} TalentsAfricains\par}

\vspace{1cm}

{\Large Plateforme Professionnelle de Mise en Valeur\par}
{\Large des Talents Africains\par}

\vspace{2cm}

\begin{tcolorbox}[colback=secondarycolor!20, colframe=secondarycolor, width=0.8\textwidth, arc=5mm]
\centering
\Large\textbf{Version 1.0}\\[0.5cm]
\normalsize
Manuel d'Utilisation Complet\\
Installation • Configuration • Utilisation\\
Administration • Maintenance
\end{tcolorbox}

\vspace{3cm}

{\large\bfseries Développé par\par}
\vspace{0.3cm}
{\LARGE\bfseries\color{primarycolor} Freddy OUEDRAOGO\par}

\vspace{1cm}

{\normalsize © 2025 - Tous droits réservés\par}
{\small Documentation mise à jour le \today\par}

\end{center}

\end{document}
